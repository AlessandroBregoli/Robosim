Nel campo dell'esplorazione multi-robotica si trovano in letteratura approcci
diversi che coprono problematiche simili. 
Troviamo ad esempio \cite{gerkey2004formal} guarda al modello
decisionale degli agenti sotto l'aspetto della task allocation, approccio da cui abbiamo tratto spunto per il nostro lavoro.
Non mancano tuttavia esempi di modelli, come in \cite{bourgault2002information},
che analizzano il problema con gli strumenti della teoria dell'informazione;
tale lavoro dà anche soluzione al problema di navigazione, stabilire la propria
posizione nell'ambiente, mentre in \cite{dudek1996taxonomy} si guarda al
problema da tutt'altra prospettiva ed il solo posizionamento relativo tra robot
vicini viene usato per condurre a termine l'esplorazione.

Vale la pena sottolineare che nel nostro lavoro non ci occupiamo della
navigazione e la posizione assoluta degli agenti sulla griglia è sempre nota con
precisione.

\cite{gerkey2004formal} Sviluppando un sistema composto da robot multipli che devono cooperare 
una delle domande chiave è: "quale robot deve eseguire che task?"; 
a questa domanda tenta di rispondere la branca della ricerca che
si occupa di multi-robot task allocation (MRTA).
Per task si intende un sottogoal necessario per ottenere l'obiettivo generale.

\subsection{Funzione da ottimizzare}
  Per trattare MRTA in un contesto di ottimiziazione è necessario decidere cosa deve essere
  ottimizzato. Idealmente il goal è di ottimizzare le performance del sistema, ma questo valore
  è spesso difficile da misurare durante l'esecuzione del sistema.
  Inoltre, quando si sceglie tra diverse possibilità, l'impatto sulle performance del sistema di ogni opzione
  solitamente non è conosciuto; quindi è necessario una stima delle performance. Viene dunque introdotto
  il concetto di "utility" definito come quel valore che:
  \begin{enumerate}
    \item Rappresenta la qualità dell'esecuzione del compito assegnato dati il
          metodo e l'equipaggiamento.
    \item Rappresenta il costo in risorse dati i requisiti spazio-temporali del
          lavoro assegnato.
  \end{enumerate}
  
\subsection{Un approccio collaborativo}
  \cite{burgard2002collaborative} L'obiettivo di un processo di esplorazione è quello di esplorare l'intero ambiente 
  nel minor tempo possibile. Questo modello utilizza un ambiente discretizzato attraverso una griglia. Durante la 
  scelta degli obiettivi per i robot vengono tenunte in considerazione le celle di frontiera (celle esplorate con
  almeno un vicino non esplorato) e si calcola una funzione di costo che è inversamente proporzionale alla distanza
  dal robot per cui si sta cercando l'obiettivo e direttamente proporzionale al numero di robot che si stanno diriginedo
  verso quella cella. Un altro dato da considerare è l'utilità di una cella di frontiera. Tale valore è
  difficile da calcolare tuttavia possiamo aspettarci che una cella obiettivo di un robot avrà un valore di utilità
  inferiore per gli altri. \\
  La selezione del target appropriato per ogni robot tiene in considerazione il costo dello spostamento verso il target
  e l'utilità della cella target. In particolare per ogni robot $i$ cerchiamo un compromesso tra il costo $V^i_t$ di
  muoversi verso la cella $t$ e l'utilità $U_t$ di $t$.
\subsection{Complessità delle mappe}
  Sono state sviluppate diverse metriche per caratterizzare la "difficoltà" di
  una mappa; la maggior parte degli approcci in letteratura si dedicano al
  problema affine di calcolare la difficoltà di un labirinto con una uscita e un
  ingresso; per dare una soluzione elegante a tale problema ad esempio,
  \cite{amancio2011concepts} e \cite{mcclendon2001complexity} si basano su una
  rappresentazione a grafo dell'ambiente e ne calcolano le proprietà; in più
  in \cite{amancio2011concepts} si fa riferimento all'``absorption time'', ovvero
  la probabilità che un cammino randomico arrivi dall'entrata all'uscita.
  Tuttavia per il nostro approccio non è sufficiente considerare la mappa un
  labirinto.
\subsection{Legge di Amdhal}
  La legge di Amdhal\cite{amdhal} in informatica è quella formula che rappresenta 
  l'aumento teorico di velocità di esecuzione derivante da una 
  parallelizzazione del problema. 
  $$S_{lantency}(s) = \frac{1}{(1-p) + \frac{p}{s}}$$
  dove:
  \begin{itemize}
    \item $S_{lantency}$: é il guadagno teorico di performance dell'intero task
    \item $s$: è il fattore che rappresenta le risorse del sistema
    \item $p$: rappresenta la proporzione tra la parte di sitema parallelizzabile e quelle non parallelizzabile
  \end{itemize}
  
  $$\begin{cases}
    S_{latency}(s) \leq \frac{1}{1 - p} \\
    \lim\limits_{s \to \infty} S_{latency}(s) = \frac{1}{1 - p}.
    \end{cases}$$
  Da questo possiamo evincere che aumentando all'infinito le risorse di parallelizzazione il sistema raggiungerà un minimo rappresentato
  dalla sua parte non parallelizzabile.
  Una possibile derivazione di questa formula è l'equazione che ricava il
  tempo teorico di esecuzione del sistema:
  $$T(s) = (1-p)T + \frac{p}{s}T$$
  
  
