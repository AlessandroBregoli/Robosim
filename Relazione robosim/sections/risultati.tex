\subsection{Misurazioni}
Per giudicare la bontà degli algoritmi sono state utilizzate varie mappe di 
diversa complessità (calcolata come in sezione \ref{mapcomplex}).
Per valutare i risultati abbiamo raccolto le seguenti misure:

\begin{itemize}
    \item Tempi medi di esplorazione (funzione del numero di robot)
    \item Efficienza media di esplorazione (funzione del tempo)
    \item Quantità media di comunicazione (funzione del tempo)
    \item La deviazione standard dei tempi ottenuti
    \item La regressione rispetto alla funzione $T(s) = (1-p)T + \frac{p}{s}T$
          dei valori dei tempi; forniamo il parametro $p$ e l'errore
          di regressione come deviazione standard dalla funzione ``fittata''.
\end{itemize}
\newcommand{\mapComplexity}[0]{}
\newcommand{\mapDescription}[0]{}
\newcommand{\nomeMappa}[0]{}
\newcommand{\pSimpleFit}[0]{}
\newcommand{\pAstarFit}[0]{}
\newcommand{\errSimpleFit}[0]{}
\newcommand{\errAstarFit}[0]{}


\foreach \s in { me, m1, m2, m3, mc} {
		%\newpage
    \input{sections/mappa_\s.tex}
    \subsection{\nomeMappa}

		\begin{figure}[H]
    		\includegraphics[width=1\textwidth]{presentaz/grafici/\s.pdf}
    		\caption{Questa mappa ha complessità \mapComplexity}
    	\end{figure}
    	
    	\mapDescription    	
    	
    	\begin{figure}[H]
          \centering
          \makebox[\textwidth][c]{
		  \subfloat[Tempi dell'algoritmo simple sulla \nomeMappa. ]{ 
			\shortstack{
		      \includegraphics[width=0.7\textwidth]{immagini/runall_\s_tempi_simple.pdf} \\
			  \small{Risultati del fit: $p = \pSimpleFit, std=\errSimpleFit$}}
			}
			\subfloat[Tempi dell'algoritmo $A*$ sulla \nomeMappa.]{
			\shortstack{
			  \includegraphics[width=0.7\textwidth]{immagini/runall_\s_tempi_astar.pdf} \\
			  \small{Risultati del fit: $p = \pAstarFit, std=\errAstarFit$}}
			}
		  }
		\end{figure}
		\begin{figure}[H]
    	  \makebox[\textwidth][c]{
		    \subfloat[Esplorazione della \nomeMappa]{
			  \includegraphics[width=0.7\textwidth]{immagini/runall_\s_espl.pdf}
			}
			\subfloat[Comunicazioni]{
			  \includegraphics[width=0.7\textwidth]{immagini/runall_\s_comun.pdf}
			}
	      }
		\end{figure}
		\begin{figure}[H]
    	  \centering
    	  \makebox[\textwidth][c]{
			\subfloat[Deviazione standard \nomeMappa]{ 
			  \includegraphics[width=0.7\textwidth]{immagini/runall_\s_std_tempi.pdf}
		    }
		  }
		\end{figure}
		
}

\subsection{Osservazioni sui risultati}
    Tutti i test sono stati svolti 5 volte per ogni possibile numero di agenti tra 1 e 20;
    tuttavia per l'algoritmo $A^*$ è stata utilizzato un valore di stubborness pari a
    0.5 mentre per l'algoritmo simple ne è stato utilizzato uno pari a 0.3; questa
    scelta deriva dal fatto che l'algoritmo simple essendo greedy entra con più facilità
    in loop con effetti quantomeno deleteri. Ciononostante è capitato, durante la
    campagna di simulazione, che alcune mappe causassero un loop nell'algoritmo simple. 
    
