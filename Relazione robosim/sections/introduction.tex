
La modellazione ad agenti è senza dubbio un metodo valido per simulare l'esplorazione
di uno spazio da parte di robot; ciò che distingue i vari approcci studiati è 
l'architettura dell'agente-robot, i metodi di interazione, il modello decisionale.

Nei modelli esistenti si cerca l'ottimalità dell'esplorazione (come tempo, come lunghezza dei percorsi), ed emergono due approcci fondamentali:
\begin{itemize}
	\item Un solo robot costoso (computazionalmente) che ricerca cerca di approssimare l'esplorazione ottimale
	\item Diversi robot poco costosi coordinati da un nodo centrale (costoso computazionalmente)
\end{itemize}
Entrambi gli approcci soffrono di una caratteristica indesiderabile: hanno un single point of failure che rende inutilizzabile l'intero modello.

Il nostro lavoro rinuncia all'ottimalità dell'esplorazione: accontentandosi di un comportamento "buono" anche se non ottimale, è possibile eliminare il coordinamento centrale, mantenere i vari robot poco costosi e ottenere robustezza di comportamento.
