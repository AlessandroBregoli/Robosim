I risultati sono generalmente coerenti con le aspettative; tuttavia particolari
conformazioni possono migliorare considerevolmente il comportamento
dell'algoritmo simple. Spesso i due algoritmi sono comparabili; tuttavia è bene
notare che la performance di $A*$ è probabilmente limitata dall'algoritmo per la 
scelta del goal, di natura greedy: infatti quest'ultimo non sfrutta
l'ottimalità del path calcolato da $A*$.

Nei risultati non è mai stata presa in considerazione la complessità degli
algoritmi come metrica, tuttavia, dato lo studio nelle sezioni \ref {complastar}
e \ref{complgreedy}, è chiaro che la scelta dell'algoritmo dovrebbe essere
influenzata dalla disponibilità di potenza computazionale dei singoli strumenti.

\subsection{Metrica di impedimento medio}
La metrica da noi sviluppata ha lo svantaggio di essere computazionalmente molto
onerosa, e di non predire fedelmente il comportamento del nostro algoritmo ``simple'';
tuttavia è anche da imputarsi dal fatto che tale algoritmo è solo parzialmente greedy,
in quanto l'accorgimento delle ``smelly cells'' gli conferisce maggior resistenza
contro i loop.

Sosteniamo tuttavia che la metrica sia effettivamente indicativa della difficoltà
di spostarsi sulla mappa con un algoritmo greedy, e più generalmente fornisca
una buona stima dell'aumento di lunghezza di un percorso dettata dalla topologia
della mappa.
\subsection{Possibili sviluppi futuri}
Sarebbe interessante caratterizzare con precisione la formazione dei
comportamenti periodici in relazione ai parametri di stubborness e della
complessità della mappa; inoltre si potrebbe ideare una metodologia di detection
e recovery dei loop, o comportamenti non del tutto deterministici volti a
renderne improbabile la formazione.

Un ambito di ulteriore sviluppo è la funzione di selezione del goal; un'euristica
che tenga maggior conto del territorio potrebbe aumentare l'efficienza
dell'esplorazione.

Se da un lato l'intero modello è stato pensato come simulazione software, ci si
può chiedere la fattibilità di un'implementazione hardware, e le sue caratteristiche;
il modello richiede un sistema di comunicazione robusto, di capacità che dipende
dal numero dei robot impiegati; inoltre la necessità di avere una comunicazione
broadcast costante tra i robot richiede un raggio di comunicazione almeno pari alla
distanza tra i 2 punti più lontani raggiungibili dagli agenti.